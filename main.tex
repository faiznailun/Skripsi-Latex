\documentclass[aspectratio=169]{beamer}
\usetheme{Madrid}
\hypersetup{pdfpagemode=FullScreen}
\usepackage[indonesian]{babel}
\usepackage{tikz}
\usetikzlibrary{shapes.geometric, arrows}
\usepackage[none]{hyphenat} %untuk menghilangkanpemenggalan kata

\title{PENERAPAN K-MEANS DAN ALGORITMA GENETIKA UNTUK
MENYELESAIKAN MTSP}
\subtitle{(Studi Kasus Pada Perjalanan Menuju Seluruh SMA di Kabupaten Probolinggo)}
\author{Muhammad Faiz Nailun Ni'am}
\institute{Universitas Nurul Jadid}
\date{\today}

\begin{document}

\newpage
\begin{titlepage}
   \begin{center}

       \textbf{PENERAPAN $K$-MEANS DAN ALGORITMA GENETIKA UNTUK MENYELESAIKAN MTSP}
       
       (STUDI KASUS PADA PERJALANAN MENUJU SELURUH SMA DI KABUPATEN PROBOLINGGO)

       \vfill
       \textbf{SKRIPSI}
       \vfill
       
       \includegraphics[width=0.4\textwidth]{logo.png}
       
       \vfill
       
       \textbf{OLEH:}\\
       \textbf{\underline{MUHAMMAD FAIZ NAILUN NI'AM}}\\
       NIM : 1842200034

       \vfill
       
       UNIVERSITAS NURUL JADID\\
       PAITON PROBOLINGGO\\
       \textbf{FAKULTAS SOSIAL DAN HUMANIORA}\\       
       PROGRAM STUDI PENDIDIKAN MATEMATIKA\\
       
       \vfill       
       
       JUNI 2022
       
   \end{center}
\end{titlepage}

% Halaman depan
\newpage
\frontmatter
\begin{center}

\textbf{PENERAPAN $K$-MEANS DAN ALGORITMA GENETIKA UNTUK MENYELESAIKAN MTSP}
       
       (Studi Kasus Pada Perjalanan Menuju Seluruh SMA di Kabupaten Probolinggo) 
       
       \vfill
       \textbf{SKRIPSI}
       \vfill
       
       DIAJUKAN KEPADA UNIVERSITAS NURUL JADID \\
       PAITON PROBOLINGGO UNTUK MENYELESAIKAN \\ SALAH SATU PERSYARATAN DALAM MENYELESAIKAN \\ PROGRAM SARJANA PENDIDIKAN MATEMATIKA
       
       \vfill       
       
       \textbf{OLEH :}\\
       \textbf{\underline{MUHAMMAD FAIZ NAILUN NI'AM}}\\
       NIM : 1842200034

       \vfill
       
       UNIVERSITAS NURUL JADID\\
       PAITON PROBOLINGGO\\
     \textbf{FAKULTAS SOSIAL DAN HUMANIORA}\\       
       PROGRAM STUDI PENDIDIKAN MATEMATIKA\\
       
       \vfill       
       
       JULI 2022
       
   \end{center}

%\newpage
%\addcontentsline{toc}{chapter}{PERSETUJUAN PEMBIMBING SKRIPSI}
%\includepdf[pages={2}]{lembar persetujuan pembimbing.pdf}

%\newpage
%\addcontentsline{toc}{chapter}{PENGESAHAN TIM PENGUJI SKRIPSI}
%\includepdf[pages={2}]{lembar persetujuan dan pengesahan.pdf}

%\newpage
%\addcontentsline{toc}{chapter}{PERNYATAAN KEASLIAN TULISAN}
%\includepdf[pages={2}]{pernyataan keaslian tulisan bermaterai.pdf} ...

\newpage
\addcontentsline{toc}{chapter}{ABSTRAK}

\chapter*{\hl{ABSTRAK}}
MTSP
\addcontentsline{toc}{chapter}{KATA PENGANTAR}
\chapter*{KATA PENGANTAR}
Segenap Puji

\newpage
%\addcontentsline{toc}{chapter}{DAFTAR ISI}
\tableofcontents

\newpage
%\addcontentsline{toc}{chapter}{DAFTAR GAMBAR}
\listoffigures

\newpage
\listoftables

% Halaman utama
\newpage
\mainmatter

\chapter{PENDAHULUAN}
\section{Latar Belakang Masalah}

Kabupaten Probobolinggo adalah salah satu dari beberapa kabupaten yang sedang berkembang di provinsi Jawa Timur. Banyak sekolah tingkat menengah yang tersebar di Kabupaten Probolinggo. Selain itu di Kabupaten Probolinggo terdapat beberapa kampus salah satunya adalah Universitas Nurul Jadid yang terletak di Kecamatan Paiton. Pada tahun-tahun sebelumnya kampus ini sering sekali mengadakan acara-acara besar seperti lomba dan olimpiade. Dalam acara-acara tersebut seringkali melakukan pendistribusian barang seperti undangan acara, pamflet, dan lain-lain kepada beberapa sekolah di Kabupaten Probolinggo. Oleh karena itu diperlukanlah sebuah pencarian rute yang efisien untuk menuju ke sekolah-sekolah tersebut agar dapat menghemat waktu dan tenaga dalam perjalanan. Permasalahan pencarian rute tersebut dalam hal ini dapat disebut dengan \textit{Traveling Salesman Problem} (TSP). Sedangkan gabungan dari beberapa permasalahan TSP disebut \textit{Multiple Traveling Salesman Problem} (MTSP).

Selama bertahun-tahun, telah banyak penelitian tentang MTSP. Berbagai metode telah banyak digunakan untuk mencari solusi dari permasalahan MTSP salah satunya adalah Algoritma Genetika (AG) dan $K$-means. Untuk melakukan proses pencarian solusi MTSP diperlukanlah proses pengklasteran (pengelompokan) terlebih dahulu, ada banyak cara untuk menggunakan AG dalam pengklasteran, terbukti bahwa metode ini dapat mengklaster data lebih cepat daripada beberapa algoritma lain yang digunakan untuk pengklasteran \cite{krishna1999genetic}. Kemampuan pengklasteran dari AG ini dimanfaatkan untuk mencari pusat klaster yang sesuai sehingga kesamaan dari klaster yang dihasilkan dioptimalkan \cite{maii2000genetic}. Ada juga yang menggunakan metode paralel untuk TSP untuk meningkatkan efisiensi seperti pada artikel \cite{li2016parallel}.

Namun, menurut artikel Zhang efisiensi AG akan menurun dengan cepat jika digunakan pada skala kota besar \cite{zhang2014parallel}, berbeda dengan algoritma $k$-means dapat mengklaster terlebih dahulu sebelum melakukan pencarian solusi dari permasalahan TSP dan menghindari persilangan tiap rute salesman (pengantar barang) \cite{inproceedings}. Penggunaan Algoritma Genetika dan dan algoritma \textit{k}-means, algoritma ini merupakan algoritma yang digunakan untuk membagi data MTSP menjadi beberapa klaster, metode ini efektif untuk menyelesaikan MTSP, selain itu juga dapat menghindari persilangan rute antar salesman seperti yang dibahas oleh Lu pada artikelnya \cite{inproceedings}. Dari gabungan semua perspektif tersebut, dalam penelitian ini, digunakanlah \textit{k}-means dan Algoritma Genetika untuk menyelesaikan kasus pembagian klaster dan pencarian rute terdekat menuju seluruh SMA di Kabupaten Probolinggo.
%\input{identifikasi masalah}
\section{Rumusan Masalah}

Berdasarkan identifikasi masalah, maka rumusan masalah yang akan dikaji dalam penelitian ini sebagai berikut:
\begin{enumerate}
    \item Bagaimana cara mencari solusi \textit{multiple traveling salesman problem} dengan $k$-means dan algoritma genetika?
    \item Bagaimana pembagian klaster dan penentuan rute terdekat menuju seluruh SMA di Kabupaten Probolinggo?
\end{enumerate}
\section{Tujuan Penelitian dan Pengembangan}

Berdasarkan rumusan masalah, tujuan dari penelitian ini yaitu untuk:
\begin{enumerate}
	\item Mengetahui cara menemukan solusi \textit{multiple traveling salesman problem} dengan $k$-means dan algoritma genetika.
	\item Menemukan solusi pembagian klaster dan penentuan rute terdekat menuju SMA di seluruh Kabupaten Probolinggo.
\end{enumerate}

\input{manfaat penelitian}
\section{Batasan Masalah Penelitian}

Berdasarkan latar belakang penelitian dan tujuan penelitian, batasan masalah dalam penelitian ini yaitu:

\begin{enumerate}
	\item MTSP pada skripsi ini menggunakan 1 titik asal dan setiap salesman akan berangkat dan kembali pada simpul kota yang sama.
	\item MTSP pada skripsi ini menggunakan $k$-means untuk pengklasteran dan algoritma genetika untuk menentukan rute terdekatnya.
	\item Titik tujuan merupakan seluruh SMA di Kabupaten Probolinggo baik negeri maupun swasta.
	\item Setiap titik tujuan diasumsikan selalu terhubung dan berjalan lurus.
	\item Tidak ada prioritas kota mana saja yang dilalui terlebih dahulu.
\end{enumerate}
%\input{definisi konsep}

\chapter{KAJIAN PUSTAKA}
\input{penelitian terdahulu}
\section{Dasar Teori}

\subsection{\textit{Multiple Traveling Salesman Problem}}

\textit{Travelling Salesman Problem} atau TSP adalah permasalahan pencarian rute paling efisien dalam sebuah perjalanan, sedangkang \textit{Multiple Travelling Salesman Problem} (MTSP) adalah gabungan dari beberapa permasalahan TSP dengan titik kumpul dan titik kembali yang sama. Menurut Al-Omeer dan Ahmed, MTSP adalah salah satu kombinatorial optimasi masalah, yang dapat didefinisikan sebagai berikut: Ada $m$ jumlah salesman yang harus melakukan perjalanan ke $n$ sejumlah kota dimulai dengan depot dan berakhir di depot yang sama \cite{al2019comparative}. Selanjutnya para salesman harus melakukan perjalanan dari satu kota ke kota lain secara terus menerus tanpa mengulang kota mana saja yang telah dilintasi oleh para salesman dan mempertimbangkan jalur terpendek selama perjalanan tersebut. Metode MTSP sebenarnya banyak sekali, namun yang digunakan dalam penelitian ini adalah algoritma genetika dan algoritma \textit{k}-means. Dalam hal ini data akan dibagi menjadi beberapa klaster terlebih dahulu sesuai dengan jumlah salesman dari perusahaan, seperti pada Gambar \ref{fig:mtsp6} dan \ref{fig:mtsp5}

\begin{figure}[H]
  \centering
  \includegraphics[width=0.5\textwidth]{Gambar/Picture1.png}
  \caption{Solusi MTSP dengan membagi menjadi 6 klaster}
  \label{fig:mtsp6}
\end{figure}

\begin{figure}[H]
  \centering
  \includegraphics[width=0.5\textwidth]{Gambar/Picture2.png}
  \caption{Solusi MTSP dengan membagi menjadi 5 klaster}
  \label{fig:mtsp5}
\end{figure}

\subsection{Algoritma}

Maulana menyebutkan dalam artikelnya algoritma adalah kumpulan perintah untuk menyelesaikan suatu masalah dan diselesaikan dengan cara sistematis, terstruktur dan logis \cite{maulana2017pembelajaran}. Algoritma digunakan untuk memcahkan permasalahan yang dialami oleh seorang pengguna program.

\subsection{Algoritma $k$-means}

$K$-Means adalah jenis metode klasifikasi tanpa pengawasan yang mempartisi item data menjadi satu atau lebih klaster \cite{agusta2007k}. $K$-Means mencoba untuk memodelkan suatu dataset ke dalam beberapa klaster sehingga item-item data dalam suatu klaster memiliki karakteristik yang sama dan memiliki karakteristik yang berbeda dengan klaster lainnya.

Menurut S Monalisa \cite{monalisa2018klasterisasi} tahapan mengklaster menggunakan algoritma \textit{k}-means adalah sebagai berikut.

\begin{enumerate}
	\item Menentukan banyak klaster
	\item Memilih beberapa \textit{centroid} secara acak sesuai banyak klaster
	\item Hitung jarak titik ke centroid dengan rumus \textit{euclidean distance} seperti Persamaan (\ref{eq:euclidean1}).
	\begin{equation}
	d_{xy}=\sqrt{\sum_{i=1}^{n}(x_i-y_i)^{2}}
	\label{eq:euclidean1}
	\end{equation}
	\item Titik-titik yang tersebar masuk ke klaster yang sama dengan titik \textit{centroid} yang paling dekat
	\item Perbarui \textit{centroid} dengan menghitung nilai rata-rata nilai pada masing-masing klaster
	\item Lakukan iterasi sebanyak mungkin dengan kembali ke tahapan 3 sampai tidak ada perubahan klaster atau perubahan nilai \textit{centroid}
\end{enumerate}


\subsection{Algoritma Genetika}

Pada artikel Hermanto disebutkan bahwa algoritma genetika adalah algoritma yang digunakan untuk mencari solusi suatu permasalahan dengan cara yang lebih alami yang terispirasi dari teori evolusi  \cite{hermawanto2003algoritma}. Dalam hal ini, algoritma genetika dapat juga digunakan untuk pencarian sebuah rute terpendek dalam sebuah kasus perjalanan.

Menurut Armanda RS \cite{armanda2016penerapan} dalam artikelnya menyampaikan penyelesaian masalah menggunakan algoritma genetika memerlukan beberapa tahapan sebagai berikut:

\begin{enumerate}
	\item Menyiapkan populasi, dalam penelitian ini yang digunakan adalah data yang telah diklaster menggunakan algoritma \textit{k}-means
	\item Melakukan reproduksi dengan \textit{crossover} dan mutasi pada pembentukan awal populasi
	\item Seleksi dengan metode elitism
	\item Menentukan nilai fitness agar mendapatkan solusi akhir yang optimal
	\item Iterasi dilakukan untuk generasi berikutnya.
\end{enumerate}

\subsection{Fitness}
Fitness adalah suatu ukuran yang dijadikan acuan untuk mengetahui baik atau tidaknya suatu individu atau bisa disebut nilai dari fungsi tujuan \cite{basuki2003strategi}. Tujuan dari penggunaan algoritma genetika adalah untuk mengoptimalkan nilai fitness dengan cara mencari nilai fitness yang paling maksimal atau minimal. Seperti dalam penelitian ini yang tujuannya adalah mencari jarak yang paling minimal maka nilai fitness nya yang dicari adalah yang paling minimal juga.

\subsection{\textit{Crossover}}
\textit{Crossover} atau persilangan adalah operator dari algoritma genetika yang melibatkan dua induk untuk membentuk kromosom baru menurut artikel \cite{hardi2014analisis}. Dalam langkah ini dilakukan dengan cara menukar sebagian gen pada kromosom induk pertama dengan gen pada kromosom induk kedua seperti pada Gambar \ref{fig:crossover}. Proses \textit{crossover} tersebut diterapkan pada setiap individu dengan probabilitas \textit{crossover} ($p_c$) yang telah ditentukan. Jika diterapkan \textit{crossover} keturunan didapatkan dari kromosom-kromosom induk. Namun jika \textit{crossover} tidak diterapkan satu induk dipilih secara random dengan $p_c$ yang sama dan diduplikasi menjadi anak.

\begin{figure}[H]
  \centering
  \includegraphics[width=0.5\textwidth]{Gambar/crossover.png}
  \caption{Proses crossover}
  \label{fig:crossover}
\end{figure}

\subsection{Mutasi}
Mutasi atau mutation adalah operator yang digunakan untuk mengubah gen-gen yang terdapat dalam kromosom. Model dalam proses ini sebagaimana yang terjadi dalam kehidupan alam \cite{rovie2014genetic} seperti pada Gambar \ref{fig:mutasi}. Dalam proses mutasi akan dibangkitkan sebuah bilangan random sebagai Probabilitas mutasi ($p_m$) yang sangat kecil. Mutasi diterapkan dengan tujuan untuk memperoleh nilai fitness yang lebih baik dari sebelumnya, dan lama-kelamaan akan menjadi solusi optimum yang diinginkan.

\begin{figure}[H]
  \centering
  \includegraphics[width=0.5\textwidth]{Gambar/mutasi.jpg}
  \caption{Proses mutasi}
  \label{fig:mutasi}
\end{figure}

\chapter{KERANGKA TEORITIK DAN PENGEMBANGAN}
%model penelitian dan pengembangan
%prosedur penelitian dan pengembangan
\section{Model Penelitian dan Pengembangan}

\textit{Research and Development (R\&D)} atau penelitian dan pengembangan adalah suatu metode penelitian yang digunakan untuk menghasilkan produk tertentu, dan menguji keefektifan produk \cite{sugiyono2013metode}. 
Berdasarkan pendapat tersebut, metode \textit{Research and Development (R\&D)} atau penelitian dan pengembangan dalam bidang pendidikan merupakan penelitian yang bertujuan untuk menghasilkan atau mengembangkan dan menvalidasi suatu produk pendidikan secara efektif.
Model penelitian dan pengembangan dalam skripsi ini melalui tahapan sebagai berikut:

\begin{enumerate}
	\item Tahap pengumpulan data, kegiatan yang dilakukan pada tahap pertama adalah peneliti mengumpulkan data. Pada tahap ini peneliti juga mencari informasi data, yaitu membaca artikel penelitian sebelumnya yang berkaitan dan juga menyiapkan alat bantu atau aplikasi yang akan digunakan untuk membantu pengolahan data. Dari tahap ini data akan dikumpulkan untuk kemudian melanjutkan ke tahapan selanjutnya.
	\item Tahap pengolahan data, pada tahap ini penulis mulai mengolah data yang telah dikumpulkan sebelumnya untuk di olah dan dari tahap ini akan dilakukan ujicoba untuk mengetahui keefektifan suatu produk.
	\item Tahap analisis, setelah mendapatkan hasil uji coba peneliti mulai menganalisis hasil, menjabarkan, serta mengevaluasinya.
	\item Tahap implementasi, pada tahap terakhir ini penelitian yang telah dievaluasi dapat digunakan dan diterapkan pada tempat penelitian.
\end{enumerate}
\section{Prosedur Penelitian dan Pengembangan}


\subsection{Data Penelitian}
    
Berdasarkan studi kasus dalam skripsi ini, data yang akan digunakan dalam penelitian ini adalah data koordinat dari seluruh SMA di Kabupaten Probolinggo. Data nama-nama sekolah dikumpulkan dari \url{https://data.sekolah-kita.net}, dan data koordinat dikumpulkan melalui aplikasi Google Earth yang dapat diunduh langsug ke dalam bentuk excel. Waktu yang diperlukan peneliti untuk mengumpulkan data dari web tersebut kurang lebih sekitar satu bulan.

\subsection{Instrumen Pendukung}
\begin{enumerate}
    \item Python
    
    Dalam penelitian ini akan digunakan bahasa pemrograman python untuk mempermudah pengerjaan. Bahasa python adalah bahasa pemrograman baru di masa sekarang, karena dalam bahasa ini lebih simple dan singkat dalam membuat program \cite{syahrudin2018input}. Bahasa pemrograman ini merupakan bahasa pemrograman yang paling mudah dipelajari dari pada bahasa pemrograman yang lain. Serta dalam bahasa pemrograman ini dapat menjalankan beberapa rumus matematika di dalamnya.
    
    \item Jupyter Notebook
    
    Jupyter Notebook adalah aplikasi web gratis yang digunakan untuk membuat dan membagikan dokumen yang memiliki kode, hasil hitungan, visualisasi, dan teks. Notebook ini juga mendukung 3 bahasa pemrograman salah satunya adalah bahasa pemrograman python. Banyak kelebihan yang disajikan dari aplikasi ini salah satunya adalah visualisasi data, mendokumentasikan kode, dan menjalankan kode dalam setiap sel.

\begin{figure}[h!]
  \centering
  \includegraphics[width=0.8\textwidth]{visualisasi jupyter.png}
  \caption{Visualisasi data menggunakan jupyter notebook}
\end{figure}

	\item Google Earth
	
	Google earth digunakan dalam penelitian ini untuk mengumpulkan koordinat lokasi seluruh SMA di Kabupaten Probolinggo. Dalam hal ini google earth dapat menandai beberapa lokasi dan mengekspor langsung kedalam bentuk excel. Data-data lokasi yang telah didownload ke dalam bentuk excel akan diproses menggunakan jupyter notebook.

\begin{figure}[h!]
  \centering
  \includegraphics[width=0.8\textwidth]{google earth.png}
  \caption{Menandai beberapa lokasi pada google earth}
\end{figure}

\begin{figure}[h!]
  \centering
  \includegraphics[width=0.8\textwidth]{ekspor excel.png}
  \caption{Mengekspor data dan menjadikannya ke format excel}
\end{figure}

\end{enumerate}

\subsection{Langkah-langkah Dalam Tahap Pengolahan Data}
\begin{enumerate}
    \item Menyiapkan data yang telah dikumpulkan sebelumnya.
    \item Selanjutnya menentukan jumlah klaster yaitu sebanyak $n$ klaster. Data yang telah dikumupulkan pada tahap ini akan dibagi menjadi beberapa klaster, metode yang digunakan algoritma \textit{k}-means.
    \item Langkah-langkah yang digunakan dalam metode \textit{k}-means adalah sebagai berikut
    \begin{enumerate}
        \item Memilih sebanyak $n$ \textit{centroid} secara acak, sesuai dengan berapa banyak salesman yang akan ditugaskan
        \item Menghitung jarak data ke \textit{centroid} dengan rumus \textit{euclidean distance}
        \begin{equation}
        d_{xy}=\sqrt{\sum_{i=1}^{n}(x_i-y_i)^{2}}
        \end{equation}
        \item Titik-titik lokasi yang tersebar merupakan klaster yang sama dengan titik \textit{centroid} paling dekat
        \item Perbarui \textit{centroid} tiap klaster yang dihasilkan dengan menghitung nilai koordinat rata-rata titik nilai pada masing-masing klaster.
        \item Iterasi dilakukan untuk generasi berikutnya sampai yaitu dengan kembali ke tahapan (b) sampai tidak ada perubahan klaster atau perubahan nilai \textit{centroid}
    \end{enumerate}
	
	\item Selanjutnya melakukan proses TSP pada setiap klaster yang telah dibagi, langkah-langkahnya adalah sebagai berikut.
	\begin{enumerate}
	    \item Membuat populasi awal secara random menggunakan data yang telah diklaster
	    \item Melakukan reproduksi dengan metode \textit{crosover} dengan peluang 0,95
	    \item Melakukan mutasi pada data dengan peluang 0,01
	    \item Selanjutnya seleksi dengan mode eliminasi
	    \item Menentukan nilai fitness agar mendapatkan solusi akhir yang optimaldengan rumus:
	    \begin{equation}
	    fitness=\frac{10000}{RMSE}
	    \end{equation}
	    \item Iterasi dilakukan dengan cara kembali ke tahapan b untuk generasi berikutnya sampai hasil yang dilakukan optimal atau mendekati optimal.
    \end{enumerate}
	\item Ketika proses diatas selesai dilakukan maka dihasilkanlah pembagian klaster dan rute terdekat tiap klaster menuju seluruh SMP di Kabupaten Probolinggo
	\item Mengevaluasi data yang dihasilkan
\end{enumerate}
%

%Agar penulisan dalam penelitian yang diusulkan lebih terarah,
%maka diperlukan sistematika penelitian.
%Terkait hal tersebut,
%sistematika penulisan dalam penelitian yang dilakukan nantinya adalah sebagai berikut.
%
%\begin{enumerate}[label=]
%
%	\item BAB I PENDAHULUAN 
%	\begin{enumerate}[label=\Alph*.]
%		\item Latar Belakang Masalah
%		\item Rumusan Masalah
%		\item Manfaat penelitian
%		\item Tujuan Penelitian dan Pengembangan
%		\item Batasan Masalah Penelitian
%	\end{enumerate}
%
%	\item BAB II KAJIAN PUSTAKA 
%	\begin{enumerate}[label=\Alph*.]
%		\item Penelitian relevan
%		\item Dasar Teori
%	\end{enumerate}
%
%	\item BAB III KERANGKA TEORITIK DAN PENGEMBANGAN 
%	\begin{enumerate}[label=\Alph*.]
%		\item Model Penelitian dan Pengembangan
%		\item Prosedur Penelitian dan Pengembangan
%	\end{enumerate}
%
%	\item BAB IV HASIL 
%	\begin{enumerate}[label=\Alph*.]
%		\item Penyajian Data Uji Coba
%		\item Analisis Data
%		\item Revisi Produk
%	\end{enumerate}
%
%	\item BAB V PENUTUP 
%	\begin{enumerate}[label=\Alph*.]
%		\item Kesimpulan
%		\item Saran
%	\end{enumerate}
%
%\end{enumerate}

\chapter{HASIL}
\section{Penyajian Data Uji Coba}

Pada penelitian ini dilakukan uji coba menggunakan data lokasi seluruh SMA di Kabupaten Probolinggo, dan dijalankan menggunakan python. Berikut adalah sajian data hasil uji coba.

\subsection{Pengambilan Data Lokasi}

Data yang digunakan adalah data koordinat lokasi yang diekspor memalui google earth. Pengujian Pengambilan data lokasi bertujuan untuk menunjukkan bahwa sistem 
mampu membaca input yang dimasukkan. Dapat dilihat pada Tabul IV.1 sebagian nama nama sekolah di kabupaten probolinggo beserta koordinat lokasinya.

\begin{table}
  \centering
  \includegraphics[width=0.8\textwidth]{data lokasi sekolah.png}
  \caption{Sebagian Data Lokasi Sekolah}
\end{table}

Setelah mendapatkan lokasi yang akan diproses, selanjutnya adalah menentukan beberapa titik centroid secara random, dalam penelitian ini akan diambil 10 centroid secara random.

\begin{table}
	\centering
	\includegraphics[width=0.8\textwidth]{centroid.png}
	\caption{Data Centroid}
\end{table}

\subsection{Proses Pengklasteran Data}

Pada tahap ini metode yang digunakan adalah metode $K-$means untuk mengklaster data
\input{analisis data}
\input{revisi produk}

\chapter{PENUTUP}
\input{kesimpulan dan saran}
% Daftar Pustaka
\newpage
\addcontentsline{toc}{chapter}{DAFTAR PUSTAKA}
\bibliographystyle{apalike}
\bibliography{library}

\newpage
\thispagestyle{empty}
\appendix
\renewcommand{\thechapter}{\arabic{chapter}}
\renewcommand{\thesection}{\thechapter.\arabic{section}}
\renewcommand{\thesubsection}{\thechapter.\arabic{section}.\arabic{subsection}}
\addcontentsline{toc}{chapter}{LAMPIRAN}
\chapter{DATASET}
\label{lampiran1}
\section{Nama dan Koordinat SMA di Kabupaten Probolinggo}

\input{Lampiran/nama sekolah}

\chapter{HASIL KLASTER}
\label{lampiran2}

\section{Pengelompokan 2 klaster}

\subsection{Klaster A}
\begin{table}[H]
\scriptsize
\centering
\begin{tabular}{lcc}
\rowcolor[HTML]{4472C4} 
{\color[HTML]{FFFFFF} \textbf{Nama   Sekolah}} & {\color[HTML]{FFFFFF} \textbf{Latitude (Sumbu X)}} & {\color[HTML]{FFFFFF} \textbf{Longitude (Sumbu Y)}} \\
\rowcolor[HTML]{D9E1F2} 
SMA ISLAM MIFTAHUL ULUM GUNUNG   GENI & -7,86 & 113,29 \\
SMA   NAZHATUT THOLIBIN               & -7,84 & 113,30 \\
\rowcolor[HTML]{D9E1F2} 
SMA NEGERI 1 DRINGU                   & -7,75 & 113,24 \\
SMA   NEGERI 1 KURIPAN                & -7,89 & 113,14 \\
\rowcolor[HTML]{D9E1F2} 
SMA NEGERI 1 SUMBERASIH               & -7,74 & 113,13 \\
SMA   TARUNA DRA. ZULAEHA             & -7,85 & 113,23 \\
\rowcolor[HTML]{D9E1F2} 
SMAN 1 BANTARAN                       & -7,84 & 113,18 \\
SMAN   1 GENDING                      & -7,81 & 113,32 \\
\rowcolor[HTML]{D9E1F2} 
SMAN 1 LECES                          & -7,87 & 113,25 \\
SMAN   1 SUKAPURA                     & -7,89 & 113,05 \\
\rowcolor[HTML]{D9E1F2} 
SMAN 1 SUMBER                         & -7,94 & 113,11 \\
SMAN   1 TONGAS                       & -7,74 & 113,10 \\
\rowcolor[HTML]{D9E1F2} 
SMAS ADDASUQI                         & -7,83 & 113,30 \\
SMAS   ASSUBHAN                       & -7,75 & 113,16 \\
\rowcolor[HTML]{D9E1F2} 
SMAS DARUL AKHLAQ                     & -7,77 & 113,14 \\
SMAS   DARUL MUKHLASHIN               & -7,85 & 113,26 \\
\rowcolor[HTML]{D9E1F2} 
SMAS DARUL ULUM                       & -7,93 & 113,33 \\
SMAS   ISLAM MIFTAHUL ARIFIN          & -7,86 & 113,18 \\
\rowcolor[HTML]{D9E1F2} 
SMAS ISLAM RADEN FATAH                & -7,84 & 113,32 \\
SMAS   ISLAM SUMBERASIH               & -7,79 & 113,17 \\
\rowcolor[HTML]{D9E1F2} 
SMAS ISLAM TAJUNG SARI                & -7,74 & 113,13 \\
SMAS   ISLAM ZAINUL HIKAM             & -7,84 & 113,22 \\
\rowcolor[HTML]{D9E1F2} 
SMAS IT KYAI SEKAR AL AMRI            & -7,84 & 113,22 \\
SMAS   MUHAMMADIYAH 3 PROBOLINGGO     & -7,82 & 113,32 \\
\rowcolor[HTML]{D9E1F2} 
SMAS NURUL HASAN                      & -7,93 & 113,31 \\
SMAS   WALI SONGO                     & -7,78 & 113,17
\end{tabular}
\end{table}

\subsection{Klaster B}
{
\scriptsize
\begin{longtable}[c]{lcc}
\rowcolor[HTML]{4472C4} 
{\color[HTML]{FFFFFF} \textbf{Nama   Sekolah}} & {\color[HTML]{FFFFFF} \textbf{Latitude (Sumbu X)}} & {\color[HTML]{FFFFFF} \textbf{Longitude (Sumbu Y)}} \\
\rowcolor[HTML]{D9E1F2} 
SMA DARUL HIKMAH                    & -7,76 & 113,43 \\
SMA   DARUT TAQWA                   & -7,79 & 113,53 \\
\rowcolor[HTML]{D9E1F2} 
SMA HAYATUL ISLAM                   & -7,88 & 113,43 \\
SMA   IRSYADUL MUBTADIIN            & -7,83 & 113,42 \\
\rowcolor[HTML]{D9E1F2} 
SMA ISLAM AR ROFIIYAH               & -7,77 & 113,41 \\
SMA   ISLAM SYARIF HIDAYATULLAH     & -7,81 & 113,51 \\
\rowcolor[HTML]{D9E1F2} 
SMA ISTIQLAL                        & -7,78 & 113,51 \\
SMA   NEGERI 1 MARON                & -7,84 & 113,36 \\
\rowcolor[HTML]{D9E1F2} 
SMA NEGERI 2 KRAKSAAN               & -7,73 & 113,46 \\
SMA   PLUS AL KHOLILIYAH            & -7,81 & 113,34 \\
\rowcolor[HTML]{D9E1F2} 
SMA SIROJUL ARIFIN                  & -7,80 & 113,42 \\
SMA   TERPADU DARUT TAUHID          & -7,81 & 113,40 \\
\rowcolor[HTML]{D9E1F2} 
SMA UNGGULAN BADRIDDUJA             & -7,75 & 113,42 \\
SMA   Zainal Abidin                 & -7,89 & 113,34 \\
\rowcolor[HTML]{D9E1F2} 
SMAN 1 BESUK                        & -7,83 & 113,50 \\
SMAN   1 GADING                     & -7,87 & 113,37 \\
\rowcolor[HTML]{D9E1F2} 
SMAN 1 KRAKSAAN                     & -7,76 & 113,42 \\
SMAN   1 KRUCIL                     & -7,94 & 113,48 \\
\rowcolor[HTML]{D9E1F2} 
SMAN 1 PAITON                       & -7,72 & 113,51 \\
SMAN   1 TIRIS                      & -7,97 & 113,40 \\
\rowcolor[HTML]{D9E1F2} 
SMAS AL HASYIMI                     & -7,79 & 113,57 \\
SMAS   AL KHAIRIYAH                 & -7,75 & 113,43 \\
\rowcolor[HTML]{D9E1F2} 
SMAS HAYATUL ISLAM                  & -7,83 & 113,43 \\
SMAS   IHYAUL IMAN                  & -7,88 & 113,45 \\
\rowcolor[HTML]{D9E1F2} 
SMAS ISLAM AR ROHMAH                & -7,93 & 113,58 \\
SMAS   ISLAM IRTIQOIYAH             & -7,79 & 113,40 \\
\rowcolor[HTML]{D9E1F2} 
SMAS ISLAM KHAIRIYAH                & -7,88 & 113,50 \\
SMAS   ISLAM MAMBAUL ULUM           & -7,85 & 113,42 \\
\rowcolor[HTML]{D9E1F2} 
SMAS ISLAM MIFTAHUL AFKAR           & -7,75 & 113,43 \\
SMAS   ISLAM MIFTAHUL ULUM JATIURIP & -7,80 & 113,39 \\
\rowcolor[HTML]{D9E1F2} 
SMAS ISLAM MIFTAHUL ULUM OPO OPO    & -7,83 & 113,41 \\
SMAS   ISLAM NURUL HUDA             & -7,94 & 113,53 \\
\rowcolor[HTML]{D9E1F2} 
SMAS ISLAM NURUR RIYADLAH           & -7,74 & 113,48 \\
SMAS   ISLAM RAUDLATUL KHAIR        & -7,79 & 113,34 \\
\rowcolor[HTML]{D9E1F2} 
SMAS ISLAM SIROJUL UMMAH            & -7,84 & 113,48 \\
SMAS   ISLAM TERPADU ULIL ALBAB     & -7,79 & 113,34 \\
\rowcolor[HTML]{D9E1F2} 
SMAS ISLAM ULUL ALBAB               & -7,86 & 113,35 \\
SMAS   MIFTAHUL HASANAIN            & -7,81 & 113,40 \\
\rowcolor[HTML]{D9E1F2} 
SMAS MUHAMMAD SHODIQ                & -7,83 & 113,37 \\
SMAS   NURUL IMAN                   & -7,86 & 113,41 \\
\rowcolor[HTML]{D9E1F2} 
SMAS NURUL JADID                    & -7,71 & 113,50 \\
SMAS   SA ADAH NIZHAMUL ISLAM       & -7,85 & 113,35 \\
\rowcolor[HTML]{D9E1F2} 
SMAS SYECH ABD QODIR ZAELANI        & -7,77 & 113,43 \\
SMAS   TAMAN MADYA                  & -7,77 & 113,41 \\
\rowcolor[HTML]{D9E1F2} 
SMAS TUNAS LUHUR                    & -7,73 & 113,52 \\
SMAS   UNGGULAN HAF-SA Z H          & -7,79 & 113,38 \\
\rowcolor[HTML]{D9E1F2} 
SMAS ZAINUL HASAN 1                 & -7,79 & 113,38 \\
SMAS   ZAINUL HASAN 2 KRUCIL        & -7,96 & 113,49 \\
\rowcolor[HTML]{D9E1F2} 
SMASI NURUL HIDAYAH                 & -7,81 & 113,40
\end{longtable}
}

\section{Pengelompokan 3 klaster}

\subsection{Klaster A}
\input{Lampiran/3a }

\section{Pengelompokan 4 klaster}
\section{Pengelompokan 5 klaster}
\section{Pengelompokan 6 klaster}
\section{Pengelompokan 7 klaster}
\section{Pengelompokan 8 klaster}
\section{Pengelompokan 9 klaster}
\section{Pengelompokan 10 klaster}

%\newpage
%\thispagestyle{empty}
%\section*{Lampiran 2 Instrumen Penelitian}

%\begin{enumerate}
%\item Jupyter Notebook

%\includegraphics[width=13cm]{instrumen2.png}

%\item Google Earth

%\includegraphics[width=13cm]{instrumen1.png}

%\end{enumerate}

\newpage
%\addcontentsline{toc}{chapter}{RIWAYAT HIDUP}
\chapter*{RIWAYAT HIDUP}
\end{document}