\chapter{PENUTUP}

\section{Kesimpulan}

Berdasarkan hasil riset dan analisis peneliti memperoleh kesimpulan sebagai berikut:

\begin{enumerate}
\item Jalur terpendek menuju seluruh SMA di Kabupaten Probolinggo dapat dihasilkan menggunakan algoritma genetika dan $k$-means dengan pembagian 7 klaster.
\item Jarak yang dihasilkan dengan pembagian klaster tersebut adalah $4,353294644$ satuan dengan urutan perjalanan sebagai berikut.

\begin{enumerate}

\item Urutan perjalanan pada klaster A:

$11\rightarrow30\rightarrow29\rightarrow47\rightarrow21\rightarrow72\rightarrow32\rightarrow56\rightarrow13\rightarrow37\rightarrow55\rightarrow36$

\item Urutan perjalanan pada klaster B:

$7\rightarrow70\rightarrow66\rightarrow28\rightarrow51\rightarrow8\rightarrow2\rightarrow34\rightarrow22$

\item Urutan perjalanan pada klaster C:

$1\rightarrow19\rightarrow73\rightarrow48\rightarrow69\rightarrow35\rightarrow46\rightarrow68\rightarrow25\rightarrow16\rightarrow5\rightarrow14\rightarrow43\rightarrow71\rightarrow53\rightarrow57$

\item Urutan perjalanan pada klaster D:

$67\rightarrow58\rightarrow23\rightarrow12\rightarrow20\rightarrow64\rightarrow39\rightarrow31\rightarrow52\rightarrow15$

\item Urutan perjalanan pada klaster E:

$26\rightarrow44\rightarrow50\rightarrow42\rightarrow74$

\item Urutan perjalanan pada klaster F:

$24\rightarrow63\rightarrow10\rightarrow59\rightarrow60\rightarrow17\rightarrow33\rightarrow9\rightarrow38\rightarrow27\rightarrow6$

\item Urutan perjalanan pada klaster G:

$40\rightarrow49\rightarrow54\rightarrow4\rightarrow41\rightarrow3\rightarrow45\rightarrow61\rightarrow18\rightarrow75\rightarrow65\rightarrow62$

\end{enumerate}

(Daftar kode sekolah dapat dilihat pada Lampiran \ref{lampiran1})

\item Penggunaan algoritma genetika akan menurun keefektifannya jika diterapkan pada skala kota besar atau semakin banyak kota yang dituju dalam sekali perjalanan. Namun hal tersebut dapat diminimalisir dengan menambahkan banyak populasi pada tahap algoritma genetika.

\end{enumerate}

\section{Saran}

Mengingat keterbatasan waktu untuk mengembangkan penelitian ini lebih jauh, maka saran kami untuk mengembangkan penelitian ini adalah:
\begin{enumerate}
    \item Mencoba algoritma lain untuk mengetahui apakah ada metode yang lebih efektif untuk mengetahui rute menuju seluruh SMA di Kabupaten Probolinggo.
    \item Untuk penelitian selanjutnya dapat menambahkan variabel waktu tempuh untuk mencari rute tercepat untuk menuju ke seluruh SMA di Kabupaten Probolinggo, karena dalam skripsi ini hanya menggunakan variabel jarak saja.
    \item Untuk penelitian selanjutnya, penghitungan jarak tidak menggunakan \textit{Euclidean distance}, tetapi menggunakan jarak real yang didapatkan di Google Maps
\end{enumerate}