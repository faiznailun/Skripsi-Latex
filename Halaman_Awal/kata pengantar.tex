\newpage
\chapter*{KATA PENGANTAR}

Segenap puji dan syukur kami panjatkan kepada Allah SWT karena atas rida, berkat, dan karunia yang diberikan oleh-Nya, sehingga peneliti berkesempatan untuk menyelesaikan penelitian skripsi ini dengan judul "Penerapan $K$-means dan Algoritma Genetika untuk Menyelesaikan MTSP (Studi Kasus pada Perjalanan Menuju Seluruh SMA di Kabupaten Probolinggo)".

Skripsi ini diajukan untuk syarat kelulusan mata kuliah Skripsi di Program Studi Pendidikan Matematika Universitas Nurul Jadid. Tidak dapat disangkal bahwa butuh usaha yang keras dalam penyelesaian penelitian dan penulisan skripsi ini. Namun skripsi ini tidak akan selesai tanpa bantuan, dukungan, bimbingan, serta do'a yang sangat berharga oleh orang-orang disekeliling kami. Oleh karena itu kami ingin mengucapkan terima kasih yang sebesar-besarnya kepada pihak yang telah memberikan dukungan dan membantu selama penyusunan skripsi ini. Terima kasih saya sampaikan kepada:

\begin{enumerate}
	\item Keluarga terutama kedua orang tua yang telah memberikan semangat dan motivasi dalam menyelesaikan penelitian ini.
	\item Bapak K.H. Abd. Hamid Wahid, M.Ag., selaku Rektor Universitas Nurul Jadid
	\item Bapak Dr. Tirmidi, selaku Dekan Fakultas Sosial dan Humaniora Universitas Nurul Jadid.
	\item Bapak Syadidul Itqan M.Pd., selaku Kaprodi Pendidikan Matematika Universitas Nurul Jadid.
	\item Bapak Nur Hamid, M.Si., Ph.D., dan Ibu Shofia Hidayah, M.Pd., selaku dosen pembimbing 1 dan 2 yang telah meluangkan waktu dan tenaganya serta memberikan bimbingan, wawasan, dan ilmu yang sangat berharga dalam menyelesaikan penelitian ini.
	\item Ibu Olief Ilmandira Ratu Farisi, S.Pd., M.Si., selaku dosen penguji proposal yang telah memberikan masukan, nasehat, dan perbaikan pada proposal penelitian yang sebelumnya telah dibuat.
	\item Seluruh teman-teman dari Prodi Matematika angkatan 2018 sebagai teman seperjuangan dalam menyelesaikan skripsi yang juga saling memberikan semangat dan dukungan dalam menyelesaikan skripsi.
	\item Semua pihak baik dosen, mahasiswa, teman, ataupun masyarakat sekitar yang secara langsung maupun tidak langsung terlibat dalam membantu peneliti menyelesaikan skripsi ini.
\end{enumerate}

Kami menyadari bahwa dalam penelitian ini masih terdapat kekurangan dan jauh dari kata sempurna. Oleh karena itu kami menerima dengan baik segala kritik dan masukan yang dapat membangun bagi kami. Kami berharap skripsi ini dapat memberi manfaat bagi banyak pihak.

\begin{flushright}
Probolinggo, 1 Juli 2022\\
Peneliti
\end{flushright}