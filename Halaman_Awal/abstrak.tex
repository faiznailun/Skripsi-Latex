\newpage
\chapter*{ABSTRAK}

Beberapa lembaga pendidikan sering kali mengadakan acara-acara besar seperti kompetisi dan olimpiade. Permasalahan pendistribusian barang  seperti poster, surat selebaran, dan undangan seringkali terjadi keterlambatan. Bagaimana menentukan rute optimum bagi beberapa pengirim barang di sebuah lembaga pendidikan akan dimodelkan secara matematis dan akan diselesaikan menggunakan metode $k$-means dan algoritma genetika. Permasalahan ini termasuk kategori \textit{Multiple Travelling Salesman Problem}. Dalam skripsi ini dilakukan pencarian rute optimum untuk menuju seluruh SMA di Kabupaten Probolinggo menggunakan $k$-means dan algoritma genetika.

\vspace{0.5cm}

\noindent \textbf{Kata Kunci:} MTSP, Algoritma Genetika, $K$-means.

\chapter*{ABSTRACT}

Some educational institutions often hold big events such as competitions and Olympics. Problems with the distribution of goods such as posters, leaflets, and invitations often result in delays. How to determine the optimum route for several shippers in an educational institution will be modeled mathematically and will be solved using the $k$-means method and genetic algorithm. This problem belongs to the \textit{Multiple Traveling Salesman Problem} category. In this thesis, the search for the optimum route to all high schools in Probolinggo Regency is conducted using $k$-means and genetic algorithms.

\vspace{0.5cm}

\noindent \textbf{Kata Kunci:} MTSP, Genetic Algorithms, $K$-means.