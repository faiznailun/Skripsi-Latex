\documentclass[aspectratio=169]{beamer}
\usetheme{Berlin}
\usepackage[indonesian]{babel}

\title{PENERAPAN K-MEANS DAN ALGORITMA GENETIKA UNTUK
MENYELESAIKAN MTSP}
\subtitle{(Studi Kasus Pada Perjalanan Menuju Seluruh SMA di Kabupaten Probolinggo)}
\author{Muhammad Faiz Nailun Ni'am}
\institute{Universitas Nurul Jadid}
\date{\today}

\begin{document}

\begin{frame}
\titlepage
\end{frame}

\begin{frame}
\frametitle{Daftar Isi}
\tableofcontents
\end{frame}

\section{Bab 1}
\subsection{Latar Belakang}
\begin{frame}
\frametitle{Latar Belakang}
Membuat presentasi dengan latex disini
\end{frame}

\subsection{Subsection 2}
\begin{frame}
\frametitle{Subsection 2}
Membuat presentasi dengan latex disini
\end{frame}

\subsection{Subsection 3}
\begin{frame}
\frametitle{Subsection 3}
Membuat presentasi dengan latex disini
\end{frame}

\section{Bab 2}
\subsection{Latar Belakang}
\begin{frame}
\frametitle{Latar Belakang}
Membuat presentasi dengan latex disini
\end{frame}

\subsection{Subsection 2}
\begin{frame}
\frametitle{Subsection 2}
Membuat presentasi dengan latex disini
\end{frame}

\subsection{Subsection 3}
\begin{frame}
\frametitle{Subsection 3}
Membuat presentasi dengan latex disini $S$
\end{frame}
\end{document}