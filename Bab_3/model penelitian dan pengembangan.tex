\section{Model Penelitian dan Pengembangan}

\textit{Research and Development (R\&D)} atau penelitian dan pengembangan adalah suatu metode penelitian yang digunakan untuk menghasilkan produk tertentu, dan menguji keefektifan produk \cite{sugiyono2013metode}. 
Berdasarkan pendapat tersebut, metode \textit{Research and Development (R\&D)} atau penelitian dan pengembangan dalam bidang pendidikan merupakan penelitian yang bertujuan untuk menghasilkan atau mengembangkan dan menvalidasi suatu produk pendidikan secara efektif.
Model penelitian dan pengembangan dalam skripsi ini melalui tahapan sebagai berikut.

\begin{enumerate}
	\item Tahap pengumpulan data, kegiatan yang dilakukan pada tahap pertama adalah peneliti mengumpulkan data. Pada tahap ini peneliti juga mencari informasi data, yaitu membaca artikel penelitian sebelumnya yang berkaitan dan juga menyiapkan alat bantu atau aplikasi yang akan digunakan untuk membantu pengolahan data. Dari tahap ini data akan dikumpulkan untuk kemudian melanjutkan ke tahapan selanjutnya.
	\item Tahap pengolahan data, pada tahap ini penulis mulai mengolah data yang telah dikumpulkan sebelumnya untuk di olah dan dari tahap ini akan dilakukan ujicoba untuk mengetahui keefektifan suatu produk.
	\item Tahap analisis, setelah mendapatkan hasil uji coba peneliti mulai menganalisis hasil, menjabarkan, serta mengevaluasinya.
	\item Tahap implementasi, pada tahap terakhir ini penelitian yang telah dievaluasi dapat digunakan dan diterapkan pada tempat penelitian.
\end{enumerate}