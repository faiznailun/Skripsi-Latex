

%Agar penulisan dalam penelitian yang diusulkan lebih terarah,
%maka diperlukan sistematika penelitian.
%Terkait hal tersebut,
%sistematika penulisan dalam penelitian yang dilakukan nantinya adalah sebagai berikut.
%
%\begin{enumerate}[label=]
%
%	\item BAB I PENDAHULUAN 
%	\begin{enumerate}[label=\Alph*.]
%		\item Latar Belakang Masalah
%		\item Rumusan Masalah
%		\item Manfaat penelitian
%		\item Tujuan Penelitian dan Pengembangan
%		\item Batasan Masalah Penelitian
%	\end{enumerate}
%
%	\item BAB II KAJIAN PUSTAKA 
%	\begin{enumerate}[label=\Alph*.]
%		\item Penelitian relevan
%		\item Dasar Teori
%	\end{enumerate}
%
%	\item BAB III KERANGKA TEORITIK DAN PENGEMBANGAN 
%	\begin{enumerate}[label=\Alph*.]
%		\item Model Penelitian dan Pengembangan
%		\item Prosedur Penelitian dan Pengembangan
%	\end{enumerate}
%
%	\item BAB IV HASIL 
%	\begin{enumerate}[label=\Alph*.]
%		\item Penyajian Data Uji Coba
%		\item Analisis Data
%		\item Revisi Produk
%	\end{enumerate}
%
%	\item BAB V PENUTUP 
%	\begin{enumerate}[label=\Alph*.]
%		\item Kesimpulan
%		\item Saran
%	\end{enumerate}
%
%\end{enumerate}