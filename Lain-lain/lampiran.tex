\newpage
\thispagestyle{empty}
\appendix
\renewcommand{\thechapter}{\arabic{chapter}}
\renewcommand{\thesection}{\thechapter.\arabic{section}}
\renewcommand{\thesubsection}{\thechapter.\arabic{section}.\arabic{subsection}}
\addcontentsline{toc}{chapter}{LAMPIRAN}
\chapter{DATASET}
\label{lampiran1}
\section{Nama dan Koordinat SMA di Kabupaten Probolinggo}

\input{Lampiran/nama sekolah}

\chapter{HASIL KLASTER}
\label{lampiran2}

\section{Pengelompokan 2 klaster}

\subsection{Klaster A}
\begin{table}[H]
\scriptsize
\centering
\begin{tabular}{lcc}
\rowcolor[HTML]{4472C4} 
{\color[HTML]{FFFFFF} \textbf{Nama   Sekolah}} & {\color[HTML]{FFFFFF} \textbf{Latitude (Sumbu X)}} & {\color[HTML]{FFFFFF} \textbf{Longitude (Sumbu Y)}} \\
\rowcolor[HTML]{D9E1F2} 
SMA ISLAM MIFTAHUL ULUM GUNUNG   GENI & -7,86 & 113,29 \\
SMA   NAZHATUT THOLIBIN               & -7,84 & 113,30 \\
\rowcolor[HTML]{D9E1F2} 
SMA NEGERI 1 DRINGU                   & -7,75 & 113,24 \\
SMA   NEGERI 1 KURIPAN                & -7,89 & 113,14 \\
\rowcolor[HTML]{D9E1F2} 
SMA NEGERI 1 SUMBERASIH               & -7,74 & 113,13 \\
SMA   TARUNA DRA. ZULAEHA             & -7,85 & 113,23 \\
\rowcolor[HTML]{D9E1F2} 
SMAN 1 BANTARAN                       & -7,84 & 113,18 \\
SMAN   1 GENDING                      & -7,81 & 113,32 \\
\rowcolor[HTML]{D9E1F2} 
SMAN 1 LECES                          & -7,87 & 113,25 \\
SMAN   1 SUKAPURA                     & -7,89 & 113,05 \\
\rowcolor[HTML]{D9E1F2} 
SMAN 1 SUMBER                         & -7,94 & 113,11 \\
SMAN   1 TONGAS                       & -7,74 & 113,10 \\
\rowcolor[HTML]{D9E1F2} 
SMAS ADDASUQI                         & -7,83 & 113,30 \\
SMAS   ASSUBHAN                       & -7,75 & 113,16 \\
\rowcolor[HTML]{D9E1F2} 
SMAS DARUL AKHLAQ                     & -7,77 & 113,14 \\
SMAS   DARUL MUKHLASHIN               & -7,85 & 113,26 \\
\rowcolor[HTML]{D9E1F2} 
SMAS DARUL ULUM                       & -7,93 & 113,33 \\
SMAS   ISLAM MIFTAHUL ARIFIN          & -7,86 & 113,18 \\
\rowcolor[HTML]{D9E1F2} 
SMAS ISLAM RADEN FATAH                & -7,84 & 113,32 \\
SMAS   ISLAM SUMBERASIH               & -7,79 & 113,17 \\
\rowcolor[HTML]{D9E1F2} 
SMAS ISLAM TAJUNG SARI                & -7,74 & 113,13 \\
SMAS   ISLAM ZAINUL HIKAM             & -7,84 & 113,22 \\
\rowcolor[HTML]{D9E1F2} 
SMAS IT KYAI SEKAR AL AMRI            & -7,84 & 113,22 \\
SMAS   MUHAMMADIYAH 3 PROBOLINGGO     & -7,82 & 113,32 \\
\rowcolor[HTML]{D9E1F2} 
SMAS NURUL HASAN                      & -7,93 & 113,31 \\
SMAS   WALI SONGO                     & -7,78 & 113,17
\end{tabular}
\end{table}

\subsection{Klaster B}
{
\scriptsize
\begin{longtable}[c]{lcc}
\rowcolor[HTML]{4472C4} 
{\color[HTML]{FFFFFF} \textbf{Nama   Sekolah}} & {\color[HTML]{FFFFFF} \textbf{Latitude (Sumbu X)}} & {\color[HTML]{FFFFFF} \textbf{Longitude (Sumbu Y)}} \\
\rowcolor[HTML]{D9E1F2} 
SMA DARUL HIKMAH                    & -7,76 & 113,43 \\
SMA   DARUT TAQWA                   & -7,79 & 113,53 \\
\rowcolor[HTML]{D9E1F2} 
SMA HAYATUL ISLAM                   & -7,88 & 113,43 \\
SMA   IRSYADUL MUBTADIIN            & -7,83 & 113,42 \\
\rowcolor[HTML]{D9E1F2} 
SMA ISLAM AR ROFIIYAH               & -7,77 & 113,41 \\
SMA   ISLAM SYARIF HIDAYATULLAH     & -7,81 & 113,51 \\
\rowcolor[HTML]{D9E1F2} 
SMA ISTIQLAL                        & -7,78 & 113,51 \\
SMA   NEGERI 1 MARON                & -7,84 & 113,36 \\
\rowcolor[HTML]{D9E1F2} 
SMA NEGERI 2 KRAKSAAN               & -7,73 & 113,46 \\
SMA   PLUS AL KHOLILIYAH            & -7,81 & 113,34 \\
\rowcolor[HTML]{D9E1F2} 
SMA SIROJUL ARIFIN                  & -7,80 & 113,42 \\
SMA   TERPADU DARUT TAUHID          & -7,81 & 113,40 \\
\rowcolor[HTML]{D9E1F2} 
SMA UNGGULAN BADRIDDUJA             & -7,75 & 113,42 \\
SMA   Zainal Abidin                 & -7,89 & 113,34 \\
\rowcolor[HTML]{D9E1F2} 
SMAN 1 BESUK                        & -7,83 & 113,50 \\
SMAN   1 GADING                     & -7,87 & 113,37 \\
\rowcolor[HTML]{D9E1F2} 
SMAN 1 KRAKSAAN                     & -7,76 & 113,42 \\
SMAN   1 KRUCIL                     & -7,94 & 113,48 \\
\rowcolor[HTML]{D9E1F2} 
SMAN 1 PAITON                       & -7,72 & 113,51 \\
SMAN   1 TIRIS                      & -7,97 & 113,40 \\
\rowcolor[HTML]{D9E1F2} 
SMAS AL HASYIMI                     & -7,79 & 113,57 \\
SMAS   AL KHAIRIYAH                 & -7,75 & 113,43 \\
\rowcolor[HTML]{D9E1F2} 
SMAS HAYATUL ISLAM                  & -7,83 & 113,43 \\
SMAS   IHYAUL IMAN                  & -7,88 & 113,45 \\
\rowcolor[HTML]{D9E1F2} 
SMAS ISLAM AR ROHMAH                & -7,93 & 113,58 \\
SMAS   ISLAM IRTIQOIYAH             & -7,79 & 113,40 \\
\rowcolor[HTML]{D9E1F2} 
SMAS ISLAM KHAIRIYAH                & -7,88 & 113,50 \\
SMAS   ISLAM MAMBAUL ULUM           & -7,85 & 113,42 \\
\rowcolor[HTML]{D9E1F2} 
SMAS ISLAM MIFTAHUL AFKAR           & -7,75 & 113,43 \\
SMAS   ISLAM MIFTAHUL ULUM JATIURIP & -7,80 & 113,39 \\
\rowcolor[HTML]{D9E1F2} 
SMAS ISLAM MIFTAHUL ULUM OPO OPO    & -7,83 & 113,41 \\
SMAS   ISLAM NURUL HUDA             & -7,94 & 113,53 \\
\rowcolor[HTML]{D9E1F2} 
SMAS ISLAM NURUR RIYADLAH           & -7,74 & 113,48 \\
SMAS   ISLAM RAUDLATUL KHAIR        & -7,79 & 113,34 \\
\rowcolor[HTML]{D9E1F2} 
SMAS ISLAM SIROJUL UMMAH            & -7,84 & 113,48 \\
SMAS   ISLAM TERPADU ULIL ALBAB     & -7,79 & 113,34 \\
\rowcolor[HTML]{D9E1F2} 
SMAS ISLAM ULUL ALBAB               & -7,86 & 113,35 \\
SMAS   MIFTAHUL HASANAIN            & -7,81 & 113,40 \\
\rowcolor[HTML]{D9E1F2} 
SMAS MUHAMMAD SHODIQ                & -7,83 & 113,37 \\
SMAS   NURUL IMAN                   & -7,86 & 113,41 \\
\rowcolor[HTML]{D9E1F2} 
SMAS NURUL JADID                    & -7,71 & 113,50 \\
SMAS   SA ADAH NIZHAMUL ISLAM       & -7,85 & 113,35 \\
\rowcolor[HTML]{D9E1F2} 
SMAS SYECH ABD QODIR ZAELANI        & -7,77 & 113,43 \\
SMAS   TAMAN MADYA                  & -7,77 & 113,41 \\
\rowcolor[HTML]{D9E1F2} 
SMAS TUNAS LUHUR                    & -7,73 & 113,52 \\
SMAS   UNGGULAN HAF-SA Z H          & -7,79 & 113,38 \\
\rowcolor[HTML]{D9E1F2} 
SMAS ZAINUL HASAN 1                 & -7,79 & 113,38 \\
SMAS   ZAINUL HASAN 2 KRUCIL        & -7,96 & 113,49 \\
\rowcolor[HTML]{D9E1F2} 
SMASI NURUL HIDAYAH                 & -7,81 & 113,40
\end{longtable}
}

\section{Pengelompokan 3 klaster}

\subsection{Klaster A}
\input{Lampiran/3a }

\section{Pengelompokan 4 klaster}
\section{Pengelompokan 5 klaster}
\section{Pengelompokan 6 klaster}
\section{Pengelompokan 7 klaster}
\section{Pengelompokan 8 klaster}
\section{Pengelompokan 9 klaster}
\section{Pengelompokan 10 klaster}

%\newpage
%\thispagestyle{empty}
%\section*{Lampiran 2 Instrumen Penelitian}

%\begin{enumerate}
%\item Jupyter Notebook

%\includegraphics[width=13cm]{instrumen2.png}

%\item Google Earth

%\includegraphics[width=13cm]{instrumen1.png}

%\end{enumerate}