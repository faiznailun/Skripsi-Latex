\documentclass[12pt,oneside]{book}
\usepackage[utf8]{inputenc}
\usepackage{graphicx} %input gambar
\usepackage{amsmath}
\usepackage{amsfonts}
\usepackage{amssymb}
\usepackage{float}
%Mengatur ke font times new rouman
\usepackage{mathptmx}

\usepackage[T1]{fontenc}
%
\usepackage{pdfpages}

% Garis tepi (margin)
\usepackage[paperheight=29.7cm,paperwidth=21cm]{geometry}
 \geometry{
 left=4cm,
 top=4cm,
 right=3cm,
 bottom=3cm,
 }

% mengatur 1 cm baris pertama paragraf
\usepackage{indentfirst}
\setlength{\parindent}{1cm}

% Mengkompres cite
\usepackage[numbers]{natbib}

% Mengatur page number
\usepackage{fancyhdr}
\pagestyle{fancy}
\fancyhead[L]{}  %mengosongkan head kiri
\fancyhead[R]{\thepage}  %memberi nomor page head kanan
\renewcommand{\headrulewidth}{0pt} %menghilangkan garis
\fancyfoot{} % menghilangkan footer


% change name Daftar Isi, Daftar Pustaka jadi section
\renewcommand{\contentsname}{DAFTAR ISI}
\renewcommand{\chaptername}{BAB}
\renewcommand{\bibname}{DAFTAR PUSTAKA}
\renewcommand{\listfigurename}{DAFTAR GAMBAR}
\renewcommand{\figurename}{Gambar}
\renewcommand{\listtablename}{DAFTAR TABEL}
\renewcommand{\tablename}{Tabel}
\renewcommand{\appendixname}{Lampiran}

% change languange dan pemenggalan kata
%\usepackage[indonesian]{babel}
\sloppy %agar teks tidak lebih saat rata kanan kiri

% numbering
\renewcommand{\thechapter}{\Roman{chapter}} %Peringkat 1
\renewcommand{\thesection}{\Alph{section}.} %Peringkat 2
\renewcommand{\thesubsection}{\arabic{subsection}.} %Peringkat 3
\renewcommand{\thefigure}{\arabic{chapter}.\arabic{figure}}%gambar
\renewcommand{\thetable}{\arabic{chapter}.\arabic{table}}%tabel
\renewcommand{\theequation}{\arabic{chapter}.\arabic{equation}}%equestion

\setcounter{secnumdepth}{3}
\renewcommand{\thesubsubsection}{\thesubsection\arabic{subsubsection}}

% baris antar paragraf
\linespread{1.5}

% mengatur title section
\usepackage{titlesec}
\titleformat{\chapter}[display]   
{\centering\normalfont\large\bfseries}
{\MakeUppercase{\chaptertitlename}\ \thechapter}{0pt}{\large}   
\titlespacing*{\chapter}{0pt}{-50pt}{40pt}
% ukuran section dan subsection
\titleformat{\section}
  {\normalfont\fontsize{12}{15}\bfseries}{\thesection}{1em}{}
\titleformat{\subsection}
  {\normalfont\fontsize{12}{15}\bfseries}{\thesubsection}{1em}{}

% untuk modif numbering
\usepackage{enumitem}

% membuat link
\usepackage{hyperref}

%\usepackage{soul} %digunakan sementara untuk meng-highlight teks yang belum diubah

\usepackage[none]{hyphenat} %untuk menghilangkanpemenggalan kata

% Perlengkapan pada table
\usepackage{colortbl} %untuk color table
\usepackage{adjustbox} %agar table pas di halaman (tidak lebih)
\usepackage{longtable} %Untuk membuat table lebih panjang
\usepackage{multirow}

% Untuk membuat subfigure, dan subtable
%\usepackage{caption}
%\usepackage{subcaption}


\usepackage{float} %digunakan agar gambar dan tabel tidak pindah, cara menggunakan dengan menambah [H]

% Untuk membuat flowchart
\usepackage{tikz}
\usetikzlibrary{shapes.geometric, arrows}

\usepackage{wrapfig} %untuk menambah wrap figure
