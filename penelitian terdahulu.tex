% sudah direvisi ibu shofia

\section{Penelitian Relevan}

Ada beberapa hasil penelitian sebelumnya yang memiliki keterkaitan dengan penelitian ini. Penelitian berjudul "Applying K-means and Genetic Algorithm for Solving MTSP" \cite{inproceedings}. Penelitian tersebut membahas tentang persilangan jalur antar tiap salesman yang dapat dihindari dengan menggukan algoritma genetika dan \textit{k}-means. Dari penelitian tersebut dihasilkan bahwa dengan penggunaan algoritma genetika dan $k$-means untuk menyelesaikan MTSP dapat meminimalisir terjadinya tabrakan antar salesman.

Penelitian kedua berjudul "Optimasi \textit{Multiple Travelling Salesman Problem} (M-TSP) Pada Penentuan Rute Optimal Penjemputan Penumpang \textit{Travel} Menggunakan Algoritme Genetika" \cite{raditya2017optimasi}. Penelitian tersebut membahas tentang permasalahan MTSP yaitu beberapa orang salesman yang akan berangkat dari kantor \textit{travel} menuju ke alamat penjemputan masing-masing penumpang. Pada permasalahan tersebut menggunakan representasi permutasi, proses reproduksi \textit{crossover} dengan \textit{one cut point crossover}, proses mutasi dengan \textit{exchange mutation}, dan proses seleksi dengan \textit{elitism selection}.

Mayuliana, N. K., Kencana, E. N., dan Harini, L. P. I. dalam artikelnya yang berjudul “Penyelesaian Multitraveling Salesman Problem dengan Algoritma Genetika” \cite{mayuliana2015penyelesaian}, mempelajari tentang kinerja algoritma genetika berdasarkan jarak minimum dan waktu pemrosesan yang diperlukan untuk 10 kali pengulangan untuk setiap kombinasi kota penjual. Artikel karangan Al-Khateeb, B., dan Yousif, M. berjudul "\textit{SOLVING MULTIPLE TRAVELING SALESMAN PROBLEM BY MEERKAT SWARM OPTIMIZATION ALGORITHM}" \cite{al2019solving} dalam artikel ini mengusulkan algoritma metaheuristik yang disebut algoritma \textit{Meerkat Swarm Optimization} (MSO) untuk memecahkan MTSP dan menjamin solusi berkualitas baik dalam waktu yang wajar untuk masalah kehidupan nyata.
