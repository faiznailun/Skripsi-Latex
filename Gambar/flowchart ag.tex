\begin{figure}[H]
\centering
\linespread{1}
%Definisi
\tikzstyle{bulat} = \tikzstyle{bulat} = [rectangle, rounded corners, minimum width=3cm, minimum height=1cm,text centered, draw=black, fill=red!30]
\tikzstyle{jajargenjang} = [trapezium, trapezium left angle=70, trapezium right angle=110, text width=2cm, minimum height=1cm, text centered, draw=black, fill=blue!30]
\tikzstyle{kotak} = [rectangle, minimum height=1cm, text width=3cm, text centered, draw=black, fill=orange!30]
\tikzstyle{kotak2} = [rectangle, minimum height=1cm, text width=5cm, text centered, draw=black, fill=orange!30]
\tikzstyle{belahketupat} = [diamond, minimum width=3cm, minimum height=1cm, text centered, draw=black, fill=green!30]
\tikzstyle{garis} = [thick,->,>=stealth]
%Gambar
\begin{tikzpicture}
\node (S) [bulat] {Mulai};
\node (in) [jajargenjang, below of=S, yshift=-0.8cm]{Dataset};
\node (pop) [kotak2, below of=in, yshift=-0.8cm] {Bangkitkan Populasi Awal};
\node (fit) [kotak, below of=pop, yshift=-0.8cm] {Hitung \textit{fitness}};
\node (sel) [kotak, right of=fit, xshift=+3.5cm] {Seleksi};
\node (cross) [kotak, below of=sel, yshift=-0.8cm] {\textit{Crossover}};
\node (mut) [kotak, below of=cross, yshift=-0.8cm] {Mutasi};
\node (opt) [belahketupat, left of=mut, xshift=-3.5cm] {\textit{fitnes} sama?};
\node (out) [jajargenjang, below of=opt, yshift=-1.7cm]{Kromosom Optimal};
\node (E) [bulat, below of=out, yshift=-0.8cm] {Selesai};
\draw [garis] (S) -- (in);
\draw [garis] (in) -- (pop);
\draw [garis] (pop) -- (fit);
\draw [garis] (fit) -- (sel);
\draw [garis] (sel) -- (cross);
\draw [garis] (cross) -- (mut);
\draw [garis] (mut) -- (opt);
\draw [garis] (opt) -- node[near start, color=black, xshift=+0.5cm]{Ya}(out);
\draw [garis] (opt) -- node[near start, color=black, xshift=+0.5cm]{Tidak}(fit);
\draw [garis] (out) -- (E);
\end{tikzpicture}
\caption{Diagram alir tahapan algoritma genetika}
\label{fig:flowag}
\end{figure}