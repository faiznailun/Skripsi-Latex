\begin{figure}[H]
\linespread{1}
\centering
%Definisi
\tikzstyle{bulat} = \tikzstyle{bulat} = [rectangle, rounded corners, minimum width=3cm, minimum height=1cm,text centered, draw=black, fill=red!30]
\tikzstyle{jajargenjang} = [trapezium, trapezium left angle=70, trapezium right angle=110, minimum height=1cm, text width=2cm, text centered, draw=black, fill=blue!30]
\tikzstyle{kotak} = [rectangle, minimum height=1cm, text width=3cm, text centered, draw=black, fill=orange!30]
\tikzstyle{belahketupat} = [diamond, text width=2cm, text centered, draw=black, fill=green!30]
\tikzstyle{garis} = [thick,->,>=stealth]

%Gambar
\begin{tikzpicture}
\node (1) [bulat] {Mulai};
\node (2) [jajargenjang, below of=1, yshift=-0.8cm]{Dataset, tentukan $n$ klaster};
\node (3) [kotak, below of=2, yshift=-1cm] {Pilih \textit{centroid} secara acak};
\node (4) [kotak, below of=3, yshift=-1cm] {Hitung \textit{fitness}};
\node (5) [kotak, below of=4, yshift=-2cm] {Pengelompokan berdasarkan \textit{fitness} terkecil};
\node (6) [kotak, right of=5, xshift=+3.5cm] {Memindahkan \textit{centroid} ke tengah area};
\node (7) [belahketupat, above of=6, yshift=+2cm]{\textit{Centroid} bergeser?};
\node (8) [jajargenjang, right of=7, xshift=+3.5cm] {Hasil $k$-means};
\node (9) [bulat, below of=8, yshift=-1.5cm] {Selesai};

\draw [garis] (1) -- (2);
\draw [garis] (2) -- (3);
\draw [garis] (3) -- (4);
\draw [garis] (4) -- (5);
\draw [garis] (5) -- (6);
\draw [garis] (6) -- (7);
\draw [garis] (7) -- node[near start, color=black, yshift=+0.5cm]{Ya}(4);
\draw [garis] (7) -- node[near start, color=black, yshift=+0.5cm]{Tidak}(8);
\draw [garis] (8) -- (9);

\end{tikzpicture}
\caption{Diagram alir tahapan $k$-means}
\label{fig:flowkmeans}
\end{figure}